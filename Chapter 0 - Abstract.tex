%%%%%%%%%%%%%%%%%%%%%%%%%%%%%%%%%%%%%%%%%%%%%%%%%
%%%%%%%%%%%% cap: Abstract %%%%%%%%%%%%%%%%%
%%%%%%%%%%%%%%%%%%%%%%%%%%%%%%%%%%%%%%%%%%%%%%%%%

\chapter*{Abstract}\label{cap:abstract_ro}
%%%%%%%%%%%% Section: Rezumat %%%%%%%%%%%%
Această lucrare de licență constă în realizarea unei platforme care interacționează cu o Interfață Creier-Calculator(ICC), mai exact cu interfața Unicorn Hybrid Black\cite{Unicorn_Technology}, pentru a facilita accesul la diverse aplicații compatibile. Scopul acestei platforme este de a face mai ușoară interacțiunea dintre o persoană cu dizabilități și un calculator în felul următor: de obicei când este nevoie de schimbarea aplicațiilor, utilizatorul cu dizabilități trebuie să aștepte după o altă persoană terță, fără dizabilități, care să schimbe aplicațiile și să recalibreze hardware-ul; prin crearea unei platforme care conține aplicațiile de care utilizatorul are nevoie și care îi permite aceluiași utilizator să schimbe de la un program la altul folosind o soluție deja existentă\cite{Unicorn_Speller}, nevoia de o astfel de persoană terță dispare, astfel dând utilizatorului libertatea de a folosi programele sale favorite fără să aibă nevoie de ajutor.
\vspace{\baselineskip}\newline
Având în vedere faptul că aplicația este destinată indivizilor cu dizabilități, va fi nevoie de asistență din partea unei persoane apte fizic pentru a adăuga noi aplicații pe platformă și pentru a elimina din cele existente în caz de nevoie. Pentru a ajuta în acest proces, o aplicație unealtă de tip installer a fost creată pentru a ajuta asistentul apt fizic în instalarea aplicațiilor ICC cu ajutorul unei interfețe grafice. Aplicațiile utilizate pentru a demonstra platforma vor fi preluate din aplicațiile create de studenții Universității de Vest Timișoara la hackathoanele br41n.io, în care aceștia au concurat cu aplicații care integrează interfețe ICC.
\vspace{\baselineskip}\newline
Astfel, această lucrare constă în două părți, soluția pentru realizarea platformei și soluția pentru realizarea aplicației pentru instalarea de aplicații noi. Pentru a crea platforma a fost folosit Unity engine pentru realizarea unei experiențe utilizator ușor de înțeles și pentru a crea legătura dintre aplicație și spellerul ICC proprietar Unicorn\cite{Unicorn_Speller} folosind limbajul de programare orientat obiect C\#. Aplicația tip installer este o aplicație dialog tip MFC scrisă în limbajul de programare C++ care se folosește de fișierele de tip batch pentru a descărca aplicațiile și pentru manipularea acestora în interiorul sistemului de fișiere Windows odată ce sunt descărcate. Deși ambele programe pot fi folosite separat unul de celălalt, folosirea acestora împreună creează o experiență mai ușoară pentru ambele categorii de utilizatori, atât cei cu dizabilități cât și cei apți fizic, astfel creând o soluție completă.


\chapter*{Abstract}\label{cap:abstract_en}
%%%%%%%%%%%% Section: Abstract %%%%%%%%%%%%
This bachelor's work consists in realising a platform that works in conjunction with a Brain-Computer Interface(BCI), specifically the Unicorn Hybrid Black BCI\cite{Unicorn_Technology}, to facilitate access to applications compatible with the aforementioned platform. The platform aims to improve accessibility for a disabled person who is using a BCI as follows: normally when changing between applications the disabled subject must wait for an able-bodied helper to change applications and recalibrate the hardware; by creating a platform that houses applications the user may need and that allows the user to change between them seamlessly using an already existing solution\cite{Unicorn_Speller} the need for the said helper is removed, thus giving the end-user more freedom in interacting with his/her favourite programs. 
\vspace{\baselineskip}\newline
Due to the fact that the application is targeted at impaired individuals, the need for an able-bodied helper will arise when it comes to adding new applications to the platform and/or removing existing ones should that need arise. An installer tool was created to aid the helper in installing the BCI applications via a graphical user interface. The apps used to demonstrate the platform will be taken (with permission) from UVT's submissions in the br41n.io hackathons, in which students competed with ideas for applications that integrate Brain-Computer interfacing.
\vspace{\baselineskip}\newline
Accordingly, this paper consists of two main parts, the platform solution itself and the installer solution. In creating the platform, the Unity engine was employed for creating an easy-to-understand user experience and for interfacing with Unicorn's proprietary BCI speller\cite{Unicorn_Speller} using the C\# programming language. The installer tool is an MFC-type dialogue application written in the C++ programming language that uses batch files to download and manipulate the application files inside of the Windows file system. While both programs can be used separately, they create an easier user experience for both user groups if used together: the impaired and the helpers, consequently creating a complete solution.
