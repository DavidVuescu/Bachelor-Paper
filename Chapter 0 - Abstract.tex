%%%%%%%%%%%%%%%%%%%%%%%%%%%%%%%%%%%%%%%%%%%%%%%%%
%%%%%%%%%%%% cap: Abstract %%%%%%%%%%%%%%%%%
%%%%%%%%%%%%%%%%%%%%%%%%%%%%%%%%%%%%%%%%%%%%%%%%%

\chapter*{Rezumat}\label{cap:abstract_ro}
\textbf{TODO: RETRANSLATE}\newline
%%%%%%%%%%%% Section: Rezumat %%%%%%%%%%%%
\hspace{\parindent} Această lucrare de licență constă în realizarea unei platforme care funcționează în conjuncție cu o Interfață de Computer Cerebral, în special BCI Unicorn Hybrid Black\cite{Unicorn_Technology}, pentru a facilita accesul la aplicațiile compatibile cu platforma menționată. Scopul platformei este de a îmbunătăți accesibilitatea pentru o persoană cu dizabilități care folosește un BCI astfel: în mod normal, când schimbă între aplicații, subiectul cu dizabilități trebuie să aștepte un asistent apt fizic să schimbe aplicațiile și să recalibreze hardware-ul. Prin crearea unei platforme care elimină necesitatea respectivului asistent, utilizatorul final beneficiază de mai multă libertate în interacțiunea cu aplicațiile sale preferate.

\vspace*{2mm}
\hspace{\parindent} Având în vedere faptul că aplicația este destinată indivizilor cu dizabilități, va fi nevoie de un asistent apt fizic pentru a adăuga noi aplicații pe platformă și pentru a elimina cele existente. Platforma va fi adaptată pentru a face această operațiune cât mai simplă posibil pentru a asigura interacțiunea minimă a asistentului menționat. Aplicațiile utilizate pentru a demonstra platforma vor fi preluate (cu permisiunea) din propunerile UVT la hackathoanele br41n.io, în care studenții au concurat cu idei pentru aplicații precum cele utilizate.

\vspace*{2mm}
\hspace{\parindent} Platforma constă din două părți principale, interfața grafică cu utilizatorul și metoda de intrare. Interfața grafică cu utilizatorul utilizează WPF \texttt{C\#} pentru grafică și Programarea Orientată pe Obiecte în limbajul \texttt{C\#} pentru logica din spatele selectorului de aplicații. Metoda de intrare utilizează BCI Speller proprietar Unicorn\cite{Unicorn_Speller} și un receptor în interiorul GUI pentru a captura intrarea utilizatorului pe o adresă UDP.


\chapter*{Abstract}\label{cap:abstract_en}
\textbf{TODO: UPDATE FORMATTING} \newline
%%%%%%%%%%%% Section: Abstract %%%%%%%%%%%%
\hspace{\parindent} This bachelor work consists in realising a platform that works in conjunction with a Brain Computer Interface, specifically the Unicorn Hybrid Black BCI\cite{Unicorn_Technology}, to facilitate access to applications compatible with aforementioned platform. The platform's aim is to improve accessibility for a disabled person that is using a BCI as follows: normally when changing between applications the disabled subject must wait for an able-bodied helper to change applications and recalibrate the hardware. By creating a platform that removes the need for said helper, the end-user benefits from more freedom in interacting with his/her favourite apps. 

\vspace*{2mm}
\hspace{\parindent} Due to the fact that the application is targeted at impaired individuals, the need for an able-bodied helper will arise when it comes to adding new applications to the platform and removing existing ones. The platform will be adapted to make this operation as straightforward as possible to ensure minimal interaction of said helper. The apps used to demonstrate the platform will be taken (with permission) from UVT's submissions in the br41n.io hackathons, in which students competed with ideas for applications such as the ones being used.

\vspace*{2mm}\textbf{TODO: UPDATE PARAGRAPH}\newline
\hspace{\parindent} \textbf{OUTDATED} The platform consists of two main parts, the graphical user interface and the input method. The graphical user interface uses WPF \texttt{C\#} for the graphics and Object Oriented Programming in the \texttt{C\#} language for the logic behind the app selector. The input method uses Unicorn's proprietary BCI Speller\cite{Unicorn_Speller} and a receiver inside the GUI to capture the user input on a UDP adress.


