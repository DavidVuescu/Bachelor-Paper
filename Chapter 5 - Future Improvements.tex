%%%%%%%%%%%%%%%%%%%%%%%%%%%%%%%%%%%%%%%%%%%%%%%%%
%%%%%%%%%%%% cap: future %%%%%%%%%%%%%%%%%
%%%%%%%%%%%%%%%%%%%%%%%%%%%%%%%%%%%%%%%%%%%%%%%%%

\chapter{Future Improvements}\label{cap:future}
Even though the project and this work are complete, there is still room for growth and the possibility of adding new features and improving existing ones. In this chapter, we will explore possible future developments for both the platform itself and the installer tool.

\section{Platform}
\subsection{Support for Different Spellers/BCIs}
In trying to create a homogeneous system of compatible applications all linked through one and documenting it in this paper, only one BCI and one speller were used. Other research-oriented platforms have numerous integrations with a plethora of Brain-Computer Interfaces available on the market. In creating this paper, however, I was limited to only one piece of hardware, the BCI provided by Unicorn. While it was more than enough to create the platform and run the necessary tests, it can't be denied that in the future, the platform can be improved upon by developing integrations with other BCI hardware.

\subsection{Integration with a Windows controller application}
The application platform provides a lot of freedom to a disabled user, but a lot more freedom can be given still with the integration of a system that controls Windows. Such an application would use a speller board for mouse movements and clicks as well as a full QWERTY keyboard for interacting with the operating system. Due to the current structure of the platform, this control system can be easily integrated with the application platform and can be a fully-fledged component of the platform.
\vspace{\baselineskip}\newline
In the context of the current implementation, such a controller wasn't implemented because no implementation of such a system can be seamless with Windows and this thesis aims to create a simple, intuitive and seamless experience for the end-user. Nevertheless, such an addition would streamline the disabled user's access to other Windows features, while the platform can serve as a link to all the user's BCI applications 



\section{Installer}
\subsection{Over-the-Air Updates}
Over-the-air updates, or OTA updates for short are updates that do not require manual intervention from any user. They enable the user to receive firmware updates seamlessly via a network. They are most commonly used in Android and IOS updates, as well as Windows and other computer updates.
\vspace{\baselineskip}\newline
While the installer is a useful tool for interfacing with batch scripts to download the platform and the applications, OTA updates could prove a wonderful improvement, so as to not have the user always checking for new versions in case new applications are made available for installation. This way, the installer can update itself by downloading the latest version of the installer binary from GitHub.

\subsection{Alternative File Hosting Platforms}
Although GitHub is one of the biggest and most used Git websites, it is not infallible. Servers can crash, the website may go down for a brief or a longer period of time, and while that happens, no alternative exists for downloading applications. While GitHub has excellent up-time records as of the writing of this thesis, outages do occur. In this sense, future development would be the addition of alternatives, so that in the case of an outage, the batch scripts can just use the alternative link(s) to download the same file. While it doesn't seem much, this would ensure increased reliability along the line 
