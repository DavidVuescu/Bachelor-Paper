%%%%%%%%%%%%%%%%%%%%%%%%%%%%%%%%%%%%%%%%%%%%%%%%%
%%%%%%%%%%%% cap: theory %%%%%%%%%%%%%%%%%
%%%%%%%%%%%%%%%%%%%%%%%%%%%%%%%%%%%%%%%%%%%%%%%%%

\chapter{Theoretical Basis}\label{cap:theory}
\hspace{\parindent} To fully understand this paper and the development of its proposed solution, it is necessary to first understand some basic concepts and theory regarding the current context. The purpose of this second chapter is to describe and explain notions regarding the subject at hand and to give examples of how they fit into the platform in order to justify their utility.


%%%%%%%%%%%%%%%%%%%%%%%%%%%%%%%%%%%%%%%%%%%%%
%%%%%%%%%%%% Section: BCIs %%%%%%%%%%%%%%%%%%
%%%%%%%%%%%%%%%%%%%%%%%%%%%%%%%%%%%%%%%%%%%%%
\section{Brain Computer Interfaces}\label{sect:bcis}
\hspace{\parindent}Brain-Computer Interfaces (BCIs) work by capturing brain signals, interpreting them, and converting these signals into instructions that are transmitted to devices to perform specific actions. They do not depend on typical neuromuscular pathways for output. BCIs primarily aim to provide functional restoration or replacement for individuals suffering from neuromuscular conditions like cerebral palsy, amyotrophic lateral sclerosis, stroke, or spinal cord injuries. Beginning from the pioneering demonstrations of spelling via electroencephalography and device control through single neurons, researchers have expanded the use of BCIs, incorporating various brain signal types for more complex controls of prosthetics, robotic arms, cursors, wheelchairs, and other devices. BCIs also hold promise in stroke rehabilitation and managing other conditions. In the future, they could even enhance the performance of medical professionals, such as surgeons. The rapidly expanding field of BCI technology, attracting interest from scientists, engineers, clinicians, and the public alike, hinges its future successes on progress in three key areas. Firstly, BCIs need hardware that can acquire signals conveniently, safely, portably, and function in all settings. Secondly, BCI systems should be validated through long-term, real-world usage studies involving individuals with severe disabilities, with the implementation of effective models for widespread dissemination. Lastly, the reliability of BCI performance needs improvement to match the dependability of natural muscle-based functions, both on a daily basis and from moment to moment. {\bfseries(TODO: REPHRASE SECTION)}

\vspace*{2mm}
\hspace{\parindent} A Brain-Computer Interface (BCI) is a technologically advanced system that captures brain signals, processes them, and transforms them into instructions that are transmitted to a device to perform a specific task. Importantly, BCIs bypass the brain's usual output pathways, such as peripheral nerves and muscles. The term BCI, by definition, strictly applies to systems that capture and utilize signals originating from the central nervous system (CNS). As such, systems activated by voice or muscle aren't classified as BCIs. Additionally, a standalone electroencephalogram (EEG) machine isn't a BCI, as it only logs brain signals without producing a response that interacts with the user's surroundings. It's a common fallacy to equate BCIs with mind-reading devices. BCIs don't extract information from unaware or non-consenting users, but rather they allow users to interact with their environment using brain signals instead of muscle movement. There is a symbiotic relationship between the user and the BCI. After a period of learning, the user generates brain signals that encode intentions, and the BCI, post its own learning period, deciphers these signals and translates them into instructions to a device to achieve the user's desired outcomes\cite{Shih_2012}. {\bfseries(TODO: REPHRASE SECTION)}