%%%%%%%%%%%%%%%%%%%%%%%%%%%%%%%%%%%%%%%%%%%%%%%%%
%%%%%%%%%%%% cap: intro %%%%%%%%%%%%%%%%%
%%%%%%%%%%%%%%%%%%%%%%%%%%%%%%%%%%%%%%%%%%%%%%%%%
\pagestyle{fancy}
\fancyhf{}% Clear header and footer
\fancyhead[RO,LE]{\thepage}% Page number on the outer (RO/LE) side of the header
\fancyhead[RE]{\nouppercase{\leftmark}}% Section name on the inner (RE) side of the header
\renewcommand{\headrulewidth}{0pt}% Remove header rule


\chapter{Introduction}\label{cap:intro}



%%%%%%%%%%%%%%%%%%%%%%%%%%%%%%%%%%%%%%%%%%%%%
%%%%%%%%%%%% Section: Motivation %%%%%%%%%%%%
%%%%%%%%%%%%%%%%%%%%%%%%%%%%%%%%%%%%%%%%%%%%%
\section{Project Motivation}\label{sect:motivation}
This project started from a wish to create a platform to store all BCI applications developed by the University's(West University of Timișoara) BCI teams during hackathons and other contests. Being part of said teams, it became apparent that most of the aforementioned applications could not be operated by a person with disabilities.
\vspace{\baselineskip}\newline
There are numerous applications developed for BCI systems, with some believing them to be the future of entertainment and the next step beyond AR and VR for an able-bodied user\cite{future_of_metaverse_BCI}. That leaves a lot to be desired when it comes to the initial purpose of the BCI, which was to be a prosthetic for the brain, with the focus being on disabled people. The research for BCIs came a long way since the first experiments which were performed on primates in the late 60s\cite{Fetz_1969}, with modern BCIs being used for both medical purposes and entertainment. 
\vspace{\baselineskip}\newline
As such, numerous applications are being developed for BCI users. The applications are however disconnected from one another such as the end-users would have a hard time adjusting from one to the other on their own. This is a major problem since the end-user is more often than not a disabled person requiring a helper. By creating this platform application I wanted to bridge the gap between applications and disabled users while taking advantage of the lack of consumer solutions when it comes to housing multiple applications in one place. I chose this topic for my bachelor's thesis in the hopes of creating such an application and in the hopes of making it easier for a disabled person to do the thing I loved doing all my life: using a computer.



%%%%%%%%%%%%%%%%%%%%%%%%%%%%%%%%%%%%%%%%%%%%%
%%%%%%%%%%%% Section: Objectives %%%%%%%%%%%%
%%%%%%%%%%%%%%%%%%%%%%%%%%%%%%%%%%%%%%%%%%%%%
\section{Objectives}\label{sect:objectives}
From a general standpoint, the main objective of the platform is described in the above section of this paper: making a cohesive application that integrates with a well-known BCI speller to deliver seamless access to different apps for a disabled user. The platform should be intuitive and easy to use, such as no explanations should be needed before using the platform. 
\vspace{\baselineskip}\ newline 
From a technical standpoint, this project aims to create a stable system capable of housing a considerable quantity of applications without failure. Due to the platform relying on multiple components (the speller, the individual applications, the speller receiver, and the platform logic itself), multiple points of failure exist. In the case of an unhandled crash, an able-bodied helper is once again needed, thus defeating the general purpose of the app. As such, error handling inside the application is the priority. 
\vspace{\baselineskip}\newline
When it comes to compatibility, the objective is to make apps integrate with the platform. This requires modifying the apps to be contained inside the platform to make them compatible. To reach this objective without compromising the aforementioned two, the platform will require as little tinkering on the application side as possible, ensuring that adding a new application is as easy as dragging and dropping its folder into the platform's own. The installation of new applications within the platform will fall into the hands of an able-bodied person, this is one of the limitations of the platform. 
\vspace{\baselineskip}\newline
All objectives hinge on the experience of the end-user and all features of this project are catered towards a disabled individual. This solution tries to prioritize accessibility and ease of use over other aspects. That being said, no testers from the target demographic have been employed in developing this solution, so it may or may not work as intended on the targeted user base. That being said, the implementation method may also work on other problems that require a platform controlled by the user.
\vspace{\baselineskip}\newline
Therefore we can conclude with 3 primary objectives: creating an easily understandable and navigable user experience, ensuring the stability of the proposed solution, and developing a way to integrate apps within the platform without the need to modify the apps themselves to fit in.



%%%%%%%%%%%%%%%%%%%%%%%%%%%%%%%%%%%%%%%%%%%%%%%%%%%%
%%%%%%%%%%%% Section: Similar Solutions %%%%%%%%%%%%
%%%%%%%%%%%%%%%%%%%%%%%%%%%%%%%%%%%%%%%%%%%%%%%%%%%%
\section{Similar Solutions}\label{sect:similar solutions}

\subsection{Unicorn Suite}
The best example of an application platform is the one that will be used in realising this thesis, the Unicorn Suite Hybrid Black. Inside the Suite, there are different applications created by g.tec, such as the Unicorn Recorder, Bandpower and UDP Interface among others, as well as different APIs intended as a means of integration with different applications. All of the said applications can be accessed through the suite, with some being behind a paywall while others being readily available as soon as the user downloads the suite\cite{Unicorn_Suite}. This solution is a notable system with high versatility and compatibility with various software solutions developed by g.tec to aid in using the Unicorn BCI.

\subsection{BCI2000}
BCI2000 is an open-source software system for BCI research that is free for any non-commercial use. The system includes software tools that can acquire and process data, present stimuli and feedback and manage the interaction with different outside devices such as robotic arms. BCI2000 is real-time, having the ability to synchronise EEG and other signals with a wide variety of biosignals and input devices such as computer mice or eye trackers. It has several modules to manage data importing and exporting in common file formats. BCI2000 operates on most Windows systems and the source code can be compiled on most Windows machines\cite{BCI2000}. The BCI2000 has different tools for users and developers, with numerous resources such as tutorials and references for further research. There are also multiple publications associated with BCI2000. Overall, the general-purpose platform created by BCI2000 is flexible and can integrate many different applications.

\subsection{OpenViBE}
OpenViBE is an open-source software platform dedicated to designing, testing and using BCIs. It is described as being software for real-time neurosciences (real-time processing of brain signals) which can be used to acquire, filter, process, classify and visualise brain signals in real-time. Since version 2.2.0, OpenViBE includes tools for offline and batch analysis of large datasets. The main use cases for OpenViBE are the medical field (assistance to disabled people, real-time biofeedback, neurofeedback, real-time diagnosis), multimedia (video games, VR), robotics and other fields related to brain-computer interfacing.

\subsection{Others}
Other platforms and systems housing BCI tools exist. Research papers such as BCI Software Platforms(Brunner et al., 2012) list 7 publicly available software platforms for brain-computer interfaces, as well as possible synergies and future developments, such as combining different components of different platforms\cite{Brunner_2012}. Other papers such as Research of Auxiliary Game Platform Based on BCI Technology(Jiang et al., 2009) present a solution for controlling video games by BCI using motor imagery, translating the motor imagery EEG into three control signals: left, right or stop; a simple game player can use it as an input device of to realise the function of operating games\cite{Jiang_2009}.
\vspace{\baselineskip}\newline
As the field of brain-computer interfacing alongside other related fields such as neuroscience, biomedical engineering and computer science advance, more solutions will be available for the public at large to experiment with. Within 60 years, the field has moved from simple artefact-sensitive EEG to making real the vision of brain-computer communication, resulting in long-term at-home usage for patients with locked-in syndrome\cite{K_bler_2019}. More innovation will bring different new implementations and similar solutions, as BCI as an assistive technology will likely be perceived as an integral part of life in the decades to come\cite{K_bler_2019}.
