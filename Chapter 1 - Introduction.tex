%%%%%%%%%%%%%%%%%%%%%%%%%%%%%%%%%%%%%%%%%%%%%%%%%
%%%%%%%%%%%% cap: intro %%%%%%%%%%%%%%%%%
%%%%%%%%%%%%%%%%%%%%%%%%%%%%%%%%%%%%%%%%%%%%%%%%%

\chapter{Introduction}\label{cap:intro}

%%%%%%%%%%%% Section: Description %%%%%%%%%%%%
\section{Description}\label{sect:description}
\hspace{\parindent} The bachelor work consists in realising a platform that works in conjunction with a Brain Computer Interface, specifically the Unicorn Hybrid Black BCI[], to facilitate access to specific applications. The platform's aim is to improve accessibility for a disabled person that is using a BCI as follows: normally when changing between applications the disabled subject must wait for an able-bodied helper to change applications and recalibrate the EEG. By creating a platform that removes the need for said helper, the end-user benefits from more freedom in interacting with his favourite apps. 

\vspace*{2mm}
\hspace{\parindent} The apps used to demonstrate the platform will be taken (with permission) from UVT's submissions in the br41n.io hackathons, in which students competed with ideas for applications such as the ones being used.


%%%%%%%%%%%% Section: Motivation %%%%%%%%%%%%
\section{Motivation}\label{sect:motivation}
\hspace{\parindent} wip


%%%%%%%%%%%% Section: Similar Solutions %%%%%%%%%%%%
\section{Similar Solutions}\label{sect:similar solutions}
\hspace{\parindent} Since the field of Brain Computer interfacing is still in its infancy, not many similar solutions exist, at least in the public space. There are multiple suites of applications available on the market, including the proprietary software Unicorn uses alongside their hardware which will also be used to build this application. All these software solutions require an able-bodied person to step in to change between applications, boards and at times even recalibrate the BCI hardware however. Thus, at the time of writing this paper, no similar solutions are known.

\hspace{\parindent} Disclaimer: While at the time of writing I am not aware of any similar solutions, this is subject to change as I further research this subject. Thereupon, as long as this paper is being redacted, I will update this section accordingly.
