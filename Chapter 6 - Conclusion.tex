%%%%%%%%%%%%%%%%%%%%%%%%%%%%%%%%%%%%%%%%%%%%%%%%%
%%%%%%%%%%%% cap: conclusion %%%%%%%%%%%%%%%%%
%%%%%%%%%%%%%%%%%%%%%%%%%%%%%%%%%%%%%%%%%%%%%%%%%

\chapter{Conclusion}\label{cap:ending}
This bachelor tackles creating a BCI platform for facilitating access to applications housed inside of the platform to disabled individuals without the need for an able-bodied helper.
\vspace{\baselineskip}\newline
The development process focused on prioritising stability and leaving room for future development to provide as much freedom as possible to the end user. Alongside the platform, an installer application was created to serve as a tool for able-bodied users and helpers to use in order to install compatible applications that interface with the platform.
\vspace{\baselineskip}\newline
The software solution that resulted from this undertaking provides a way to minimise the constant need for helping hands when dealing with Brain-Computer Interfacing in impaired individuals. As such, an impaired user can put on a BCI cap and use applications like an unimpaired one would, with the ability to easily switch between applications, close them at will and open new ones. The platform supports any application designed to work with the Unicorn Speller as such that with future app developments, it could be used to help someone that has never been able to use a computer before to do just that. 
\vspace{\baselineskip}\newline
The open-endedness of the application allows other developers to build upon its foundations and expand it, bridging the gap between disability and ability and bringing freedom to impaired individuals using BCI technology.